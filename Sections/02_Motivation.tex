\section{Motivation}
\label{sec:Motivation}

Why study transfer learning? We can offer two practical reasons and one theoretical reason.

\subsection{Transfer Learning: Theoretical Implications}

One theoretical reason is to study the nature of
generality in software engineering.  Professional
societies assumesuch generalities exist when they
offer lists of supposedly general ``best practices''
such as the IEEE 1012 standard for software
verification~\cite{1012}.  Endres & Rombach offer dozens
of lessons of software engineering~\cite{endres03}.  Many
other widely-cited researchers dothe same; e.g.
Glass~\cite{glass02}; Jones~\cite{jones10}; Boehm~\cite{hoehm00b}.  Budgen
&Kitchenham seek to reorganize SE research using
general conclusions drawn from a larger number of
studies~\cite{budgen06,budgen09}.

Given the constant pace of change within SE, how can
any conclusion stand the test of time?  Worse still,
numerous {\em local learning} results show that we
should mistrust general conclusions (made over a
wide population of projects) since they may not hold
for projects. For example, Yang et
al.~\cite{yang11}, Bettenburg et
al.~\cite{betten14}, and Menzies et al.~\cite{me12d}
all explore the generation of models using {\em all}
data versus {\em local} samples that more specific
to particular test cases. They all report the same
thing: better models (sometimes with much lower
variance in their predictions) are generated from
local information.

Hess


Prior attempts at offering general conclusions for SE~\cite{}
